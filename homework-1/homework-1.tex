\documentclass[12pt]{article}

\usepackage[margin=0.75in, top=1in, bottom=1in, a4paper]{geometry}
\usepackage{amsmath}
\usepackage{amssymb}
\usepackage{amsthm}
\usepackage{pgfplots}
\usepackage{mathtools}
\usepackage{booktabs}
\usepackage{indentfirst}

\newcommand{\mo}[1]{\lvert #1 \rvert}
\newcommand{\mos}[1]{\lvert #1 \rvert^2}
\newcommand{\mov}[1]{\lvert \vec{#1} \rvert}
\newcommand{\RR}{\mathbb{R}}
\newcommand{\p}{\partial}
\newcommand{\iv}[1]{\langle #1 \rangle}
\newcommand{\adj}{\text{adj}}
\newcommand{\pmax}{\text{pmax}}
\newcommand*\diff{\mathop{}\!\mathrm{d}}

\theoremstyle{definition}
\newtheorem{prop}{Proposition}[section]
\newtheorem{prob}[prop]{Problem}
\newtheorem{ex}{Exercise}

\title{\vspace{-2.0cm}Convex Optimization Homework 1}
\author{Linxuan Ma}

\begin{document}
	\maketitle
	
	\section{Convex Sets}
	
	\begin{prop}
		A polyhedron $\{x \in \RR^n : Ax \leq b\}$ for $A \in \RR^{m \times n}$ and $b \in \RR^m$ is both convex and closed.
	\end{prop}
	\begin{proof}
		The polyhedron $P = \{x \in \RR^n : Ax \leq b\}$ can be described as the intersection of $n(P)$ half-spaces, denoted by set $H$:
		\begin{gather*}
			A_{1,1}x_1 + A_{1,2}x_2 + \dots + A_{1,n}x_n \leq  b_1 \\
			\dots \\
			A_{m,1}x_1 + A_{m,2}x_2 + \dots + A_{m,n}x_n \leq  b_m
		\end{gather*}
		
		$\forall a, b \in P$, both $a$ and $b$ are in all of the above half-spaces (by definition of intersection $P \in H_i$ for all $i$), and the line connecting them can be obtained via their convex combination $z(t) = tx + (1-t)y$ for $t \in [0, 1]$. Half-spaces are convex, and therefore for any half-space $H$, $x, y \in H \Rightarrow z(t) \in H$. Since the above is satisfied for all above half-spaces,  $z(t) \in P$ (definition of intersection). Therefore, $P$ satisfies that $x, y \in P \Rightarrow tx + (1-t)y \in P$, proving that $P$ is indeed convex.
		
		The closeness of $P$ can be proven by considering the open set complement $\overline{H_i}$ of every half-space $H_i$. $\bigcup_i \overline{H_i}$ is an open set, and $P = \bigcup_i H_i$ is its complement. Therefore $P$ is a closed set (definition of closeness).
	\end{proof}
	
	\begin{prop}
		Given a set of convex sets $S$, the intersection $\bigcap_i S_i$ is convex. Similarly, given a set of closed sets $S$, the intersection $\bigcap_i S_i$ is closed.
	\end{prop}
	
	\begin{proof}
		The intersection $P = \bigcap_i S_i$ is a subset of every $S_i$ (definition of intersection). For all $x, y \in P, t \in [0, 1]$, $tx + (1-t)y \in S_i$ for all $i$ (definition of convex set). The line $tx + (1-t)y$ is contained in every $S_i$, therefore it is contained in the intersection $P$ of all $S_i$ (definition of intersection).
		
		(Same as \textit{1.1}) The closeness of $P$ can be proven by considering the open set complement $\overline{H_i}$ of every half-space $H_i$. $\bigcup_i \overline{H_i}$ is an open set, and $P = \bigcup_i H_i$ is its complement. Therefore $P$ is a closed set (definition of closeness).
	\end{proof}
	
	\begin{prob}
		Give an example of a closed set in $\RR^2$ whose convex set is not closed.
	\end{prob}
	
	\begin{proof}
		The set of $(x, y)$ satisfying
		\begin{gather*}
			y \geq \frac{1}{x^2 + c}
		\end{gather*}
		where $c \in \RR_+$ is a closed set, while its convex hull is any $(x_c, y_c)$ satisfying $y > 0$ (an open set).
	\end{proof}
	
	\begin{prop}
		Consider matrix $A \in \RR^{m \times n}$. If set $S \subseteq \RR^m$ is convex, then its pre-image of $S$ under $A$ is also convex. The same applies for closeness.
	\end{prop}
	
	\begin{proof}
		Let $P$ be the pre-image of $S$ under $A$. Consider points $x, y \in P$ and the image of their convex combination under $A$:
		\begin{align*}
			&A(tx + (1 - t)y) \\
			\iff &A(tx) + A((1 - t)y) \\
			\iff &t*Ax + (1-t) * Ay
		\end{align*}
		
		$t*Ax + (1-t) * Ay \in S$ (definition of convexity) as $Ax$ and $Ay$ are both contained in $S$ (definition of pre-image). Therefore, $A(tx + (1 - t)y) \in S \Rightarrow tx + (1 - t)y \in P$.
	\end{proof}
	
	\begin{prop}
		Consider matrix $A \in \RR^{m \times n}$. If set $S \subseteq \RR^n$ is convex, then its image under $A$ is also convex.
	\end{prop}
	
	\begin{proof}
		Consider points $x, y \in A(S)$ and some points $x_p, y_p \in S$ such that $A(x_p) = x$, $A(y_p) = y$. Since $S$ is convex, then the convex combination of $x$ and $y$ can be is equivalent to:
		\begin{align*}
			&tx + (1 - t)y \\
			\iff &t * A(x_p) + (1 - t) * A(y_p) \\
			\iff &A(tx_p) + A((1 - t)y_p) \\
			\iff &A(tx_p + (1 - t)y_p)
		\end{align*}
		
		Since $S$ is convex, $tx_p + (1 - t)y_p \in S$. Therefore, $A(tx_p + (1 - t)y_p) \in A(S)$, and equivalently, $tx + (1 - t)y \in A(S)$.
	\end{proof}
	
	\begin{prop}
		Given an example of a matrix $A \in \RR^{m \times n}$ and a set $S \subseteq \RR^n$ that is closed and convex but such that $A(S)$ is not closed.
	\end{prop}
	
	\begin{proof}
		Consider the matrix $A: \RR^2 \to \RR^2$ that discards the $x$-component:
		\begin{equation*}
		\begin{bmatrix}
			0&0\\0&1	
		\end{bmatrix}
		\end{equation*}
	\end{proof}
	
	The image of the set $S = \{(x, y):y \geq 0.5^x\}$ is both closed and convex, yet its image under $A$ is $0 \times (0, \infty]$, which is not closed.
	
\end{document}
